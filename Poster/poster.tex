% Unofficial University of Cambridge Poster Template
% https://github.com/andiac/gemini-cam
% a fork of https://github.com/anishathalye/gemini
% also refer to https://github.com/k4rtik/uchicago-poster

\documentclass[final]{beamer} 

% ====================
% Packages
% ====================

\usepackage[T1]{fontenc}
\usepackage{lmodern}
\usepackage[orientation=portrait,size=a2,scale=1.15]{beamerposter}
\usetheme{gemini}
\usecolortheme{nott}
\usepackage{graphicx}
\usepackage{booktabs}
\usepackage{tikz}
\usepackage{pgfplots}
\pgfplotsset{compat=1.14}
\usepackage{anyfontsize}


% ====================
% Lengths
% ====================

% If you have N columns, choose \sepwidth and \colwidth such that
% (N+1)*\sepwidth + N*\colwidth = \paperwidth
\newlength{\sepwidth}
\newlength{\colwidth}
\setlength{\sepwidth}{0.025\paperwidth}
\setlength{\colwidth}{0.45\paperwidth}

\newcommand{\separatorcolumn}{\begin{column}{\sepwidth}\end{column}}

% ====================
% Title
% ====================

\title{Filmcast: Algoritmos para la selección óptima de actores}

\author{Alaix P. Andrés \and Alvarado B. Ludwig \and Chocontá R. Daniela \and Martínez G. Juan  }


% ====================
% Footer (optional)
% ====================

\footercontent{
   Facultad de Ciencias Naturales e Ingeniería \hfill
  XXVIII Feria de Proyectos de Aula \hfill
  Minería de Datos \& Aplicaciones Móviles}
% (can be left out to remove footer)


% ====================
% Logo (optional)
% ====================

% use this to include logos on the left and/or right side of the header:
\logoright{\includegraphics[height=2.5cm]{logos/utadeo-logo.png}}




% ====================
% Body
% ====================

\begin{document}

% Refer to https://github.com/k4rtik/uchicago-poster
% logo: https://www.cam.ac.uk/brand-resources/about-the-logo/logo-downloads
% \addtobeamertemplate{headline}{}
% {
%     \begin{tikzpicture}[remember picture,overlay]
%       \node [anchor=north west, inner sep=3cm] at ([xshift=-2.5cm,yshift=1.75cm]current page.north west)
%       {\includegraphics[height=7cm]{logos/unott-logo.eps}}; 
%     \end{tikzpicture}
% }

\begin{frame}[t]
\begin{columns}[t]
\separatorcolumn

\begin{column}{\colwidth}

  \begin{block}{RESUMEN}

    Resumen de este increíble proyecto y tal.

  \end{block}

  \begin{block}{OBJETIVOS}

    Objetivos muy importantes.

    \begin{itemize}
      \item \textbf{Objetivo 1}
      \item \textbf{Objetivo 2}
      \item \textbf{Objetivo 3}
    \end{itemize}

    Cosas muy importantes, hmm...

  \end{block}

  \begin{alertblock}{METODOLOGÍA}

    Cómo hicimos este proyecto? No sé

  \end{alertblock}

 \begin{block}{RESULTADOS}

   \begin{figure}[H]
     \centering
     \includegraphics[width=\linewidth]{logos/webapp.png}
   \end{figure}

   \begin{figure}[H]
     \centering
     \includegraphics[width=\linewidth]{logos/webapp2.png}
   \end{figure}     

  \end{block}

\end{column}

\separatorcolumn

\begin{column}{\colwidth}

  \begin{block}{CONCLUSIONES}

    Conclusiones de esta cosa que llamamos proyecto.

  \end{block}

  \begin{block}{REFERENCIAS BIBLIOGRAFÍCAS}

    Bibliografía (BibTeX, btw).

  \end{block}

  

\end{column}
\separatorcolumn



\end{columns}
\end{frame}

\end{document}
